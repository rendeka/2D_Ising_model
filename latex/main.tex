%%%%%%%%%%%%%%%%%%%%%%%%%%%%%%%%%%%%%%%%%
% Wenneker Assignment
% LaTeX Template
% Version 2.0 (12/1/2019)
%
% This template originates from:
% http://www.LaTeXTemplates.com
%
% Authors:
% Vel (vel@LaTeXTemplates.com)
% Frits Wenneker
%
% License:
% CC BY-NC-SA 3.0 (http://creativecommons.org/licenses/by-nc-sa/3.0/)
% 
%%%%%%%%%%%%%%%%%%%%%%%%%%%%%%%%%%%%%%%%%

%----------------------------------------------------------------------------------------
%	PACKAGES AND OTHER DOCUMENT CONFIGURATIONS
%----------------------------------------------------------------------------------------

\documentclass[11pt]{scrartcl} % Font size
\input{structure.tex} % Include the file specifying the document structure and custom commands

%----------------------------------------------------------------------------------------
%	TITLE SECTION
%----------------------------------------------------------------------------------------

\title{	
	\normalfont\normalsize
	\textsc{Faculty of Mathematics and Physics}\\ % Your university, school and/or department name(s)
	\vspace{25pt} % Whitespace
	\rule{\linewidth}{0.5pt}\\ % Thin top horizontal rule
	\vspace{20pt} % Whitespace
	{\huge 2D Ising model}\\ % The assignment title
	\vspace{12pt} % Whitespace
	\rule{\linewidth}{2pt}\\ % Thick bottom horizontal rule
	\vspace{12pt} % Whitespace
}

\author{\LARGE Andrej Rendek} % Your name

\date{\normalsize\today} % Today's date (\today) or a custom date

\begin{document}
\maketitle

\section{Simulation parameters}

For the grid with a length $n = 16$ we have made $N\_INDEP = 1000$ independent simulations, for grids $n = 32$ and $n = 64$ we used just $N\_INDEP = 200$. In each case we used $N\_TSWEEP = 50$ swipes for thermalization and $N\_SWEEP = 500$ swipes for simulation, updating tracked data after each swipe after thermalization.

\section{Results} 

\begin{figure}[h] 
	\centering
	\includegraphics[width=0.8\textwidth]{magnetization.png} 
	\caption{Magnetization in dependence on temperature.}
	\label{fig:magnetization}
\end{figure}


\begin{figure}[h] 
	\centering
	\includegraphics[width=0.8\textwidth]{susceptibility.png} 
	\caption{Magnetic susceptibility in dependence on temperature.}
	\label{fig:susceptibility}
\end{figure}

\begin{figure}[h] 
	\centering
	\includegraphics[width=0.8\textwidth]{energy.png} 
	\caption{Internal energy in dependence on temperature.}
	\label{fig:energy}
\end{figure}

\begin{figure}[h] 
	\centering
	\includegraphics[width=0.8\textwidth]{specific-heat.png} 
	\caption{Specific heat in dependence on temperature.}
	\label{fig:specific-heat}
\end{figure}

The best estimation of critical temperature $T_c$ corresponding to the maximum of specific heat in \ref{fig:specific-heat} is:

\begin{equation*}
	T_c = \SI{2.307}{\kelvin}.
\end{equation*}

\end{document}
